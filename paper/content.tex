%!TEX root = main.tex

% -------------
\section{Introduction}
\label{sec:introduction}

COVID-19 caused global pandemic mode to be enabled in 2019 ~\cite{covid-wiki} ~\cite{arxiv-2009.12325}.

The first case was identified in Wuhan, China, in December 2019.

SARS-CoV-2 is a cause which is defined as severe acute respiratory syndrome coronavirus 2.

COVID-19 is an effect of coronavirus disease 2019.

Previously it was discovered 4 pandemics for the last century: in 1918, 1957, 1968 and 2009 ~\cite{cdc-pastpandemics}.

The research is dedicated to the fifth pandemic mode of century - COVID-19.

Technical part of research consists of applying following computer science methods over the \dbName{} dataset.

\begin{itemize}
    \item attribute selection
    \item aggregation
    \item pattern recognition
    \item mean feature classification
    \item linear and isotonic regression models
    \item density map plotting
    \item k-means clustering
\end{itemize}


% -------------
\section{Normalization}
\label{sec:normalization}

Attribute Selection helps to clean data.

Dataset consists of 15 unique properties.

After applying method of attribute selection only 3 properties left.

Total number of confirmed instances with submission date allows to gain more accurate numbers for predictor model.

Other properties have missed or inconsistent\ref{tab:selected_attributes} values.

\begin{table}[h]
    \centering
    \begin{tabular}{l|l|r}
    \toprule
    Property               & Description                           & Type \\\midrule
    submission\_date        & Date of counts                        & Date \& Time \\
    conf\_cases             & Total confirmed cases                 &  Number  \\
    conf\_death             & Total number of confirmed deaths      &  Number  \\
    \bottomrule
    \end{tabular}
    \caption{Selected dataset attributes ~\cite{cdc-dataset}}
    \label{tab:selected_attributes}
\end{table}


The dataset has \dbTotalInstances{} number of instances to work with.

In our research actual range was used - 04/15/20 - 12/27/20 with 0 as starting point at 04/01/20.

Total discovered points to works with is \textit{55}.

Selected attributes were grouped by 1, 5, 10, 15, 20, 25 days of each month with have medium values.

Numbers have cumulative feature which is shown in their increasing exponential progression~\ref{tab:source_numbers}.

% Numbers are close to estimated and proper methodology requires advanced data ~\cite{natstcopr-methv2}.

Recovery formula~\ref{eq:recovery} has been introduced to increase accuracy of research and observe deviations.

\begin{equation}
     Recovery = Case - Death
    \label{eq:recovery}
\end{equation}

\begin{figure}[h]
    \includegraphics*[width=4cm, height=4cm]{../src/output/death.png}
    \includegraphics*[width=4cm, height=4cm]{../src/output/recovery.png}
    \caption{Raw Death/Recovery Rates}
    \label{fig:raw_data}
\end{figure}

The slope of first image has sharp slope in the beginning and linear slope form in the tail.

Meanwhile beginning of another image has linear line slope and the tail has sharp curve form.

Both functions growth bounded above by {\textit{n log n} }~\ref{eq:asymptotic_nlogn} and below by $\sqrt{n}$~\ref{eq:asymptotic_n} asymptotically.

\begin{equation}
    f(x) \in \O O(n log n)
    \label{eq:asymptotic_nlogn}
\end{equation}

\begin{equation}
    f(x) \in \Omega (\sqrt{n})
    \label{eq:asymptotic_n}
\end{equation}

% AN Normalization
To get rid off increasing progression that the slope is too steep the shatter semi-curve segmentation algorithm was developed~\ref{fig:raw_data}~\ref{alg:shatter_segmentation}.

\begin{algorithm}[h]
    \begin{algorithmic}
        \Function{ShatterSegmentation}{DictionarySet $F$}
	    \State $Result \gets F$\Comment{Copy Data}
            \For{($i=0$; $\;i < \dbTotalInstances{} $; $\;i$++)}
                \State $A \gets F_i$\Comment{Current Day}
                \State $B \gets F_{i+1}$\Comment{Next Day}
                \State $Result_i \gets A- B $\Comment{Difference}
            \EndFor
            \State \Return $Result $
        \EndFunction
    \end{algorithmic}
    \caption{Shatter semi-curve segmentation algorithm}
    \label{alg:shatter_segmentation}
\end{algorithm}

Applied segmented algorithm results~\ref{fig:segmentated_data}.

\begin{figure}[h]
    \includegraphics*[width=4cm, height=4cm]{../src/output/death-diff.png}
    \includegraphics*[width=4cm, height=4cm]{../src/output/recovery-diff.png}
    \caption{Segmented Death/Recovery Rates}
    \label{fig:segmentated_data}
\end{figure}

The first image has strong noticeable wave in the beginning of the line.

The following distribution of the first image is almost normal except the rising wave in the tail.

The second image has 3 noticeable waves.

First and second waves are almost similar except the biggest wave size in the tail.

% todo - draw interpolation lines over

Alternatively, square-normalization approach was applied to reduce overall value of Y-axis and to preserve curve angles as a vector ratio scale~\ref{fig:normalized_segmentated_data}.

\begin{figure}[h]
    \includegraphics*[width=4cm, height=4cm]{../src/output/death-diff-sqrt-norm.png}
    \includegraphics*[width=4cm, height=4cm]{../src/output/recovery-diff-sqrt-norm.png}
    \caption{Square-normalization Death/Recovery Rates}
    \label{fig:normalized_segmentated_data}
\end{figure}

Results of the applied shatter segmentation~\ref{alg:shatter_segmentation} and square-normalization algorithms. %~\ref{graph:square_normalization}.

Waves became less perceptible and more smoothly because of angles acute forms.

% describe that curve gained noise because of normalization

% mention that low death rate was in middle of june

% two covid cases waves may and aug

Furthermore, Segmentation algorithm~\ref{alg:shatter_segmentation} was applied to population dataset~\ref{fig:raw_and_segmentated_population}.

\begin{figure}[h]
    \includegraphics*[width=4cm, height=4cm]{../src/output/population.png}
    \includegraphics*[width=4cm, height=4cm]{../src/output/population-diff.png}
    \caption{Raw and Segmented population of USA 2020}
    \label{fig:raw_and_segmentated_population}
\end{figure}

The first function bounded above by {\textit{n log n} }~\ref{eq:asymptotic_nlogn} and below by $\sqrt{n}$~\ref{eq:asymptotic_n} asymptotically.

The second image has noticeable wave in the middle of image.

% todo - finish statistical part

% Gained statistical properties from curves \ref{table:graph_stats}.

% https://en.wikipedia.org/wiki/Big_O_notation#/media/File:Comparison_computational_complexity.svg

% https://en.wikipedia.org/wiki/Pearson_correlation_coefficient

% http://ctan.math.utah.edu/ctan/tex-archive/macros/latex/contrib/algorithm2e/doc/algorithm2e.pdf



% -------------
% Results
% -------------
\section{Results}
\label{sec:results}


% CORRELATION
\subsection{COVID-19 impact on population of USA}
\label{subsec:covid-19-impact-on-population-of-usa}

\begin{figure}[h]
    \includegraphics*[width=9cm, height=5cm]{../src/output/cases_and_death_diff.png}
    \caption{Case and Death Rate}
    \label{fig:cases_death_corr}
\end{figure}

Death rate curve has only one small wave in the beginning of the line.

Case rate has 3 waves with the strongest wave in the tail of the curve.

The death rate function has linear form with constant ~\ref{eq:asymptotic_one} asymptotic.

\begin{equation}
    f(x) \in \O O(1)
    \label{eq:asymptotic_one}
\end{equation}

Only small impact on population is noticeable in the beginning of the timeline~\ref{fig:cases_death_corr}.

\begin{figure}[h]
    \includegraphics*[width=9cm, height=5cm]{../src/output/population_and_death_diff.png}
    \caption{Population and Death Rate Correlation}
    \label{fig:pop_death_corr}
\end{figure}

The death rate function~\ref{fig:pop_death_corr} has 1 wave in the beginning of the timeline.

Same as on the previous figure, the death rate function~\ref{fig:pop_death_corr}  has linear form with constant ~\ref{eq:asymptotic_one} asymptotic.

The population rate has 1 wave in the middle of the timeline.

It can be assumed because of that the coronavirus death rate has no effect on population of the USA, otherwise the death rate suppose to have a wave in the middle.

% todo: sravit civfu i viyasnit otnoshenie (ratio)

% On year 2020 World Population reported - \textit{7,794,798,739} with density \textit{52} person per square kilometer ~\cite{worldometer-population}.



% REGRESSION
\subsection{Regressions Tendency}
\label{subsec:regressions-tendency}

% Public algorithms helps researchers to compare algorithms and to observe their accuracy.

Multiple regression models were used in research to cross-validate estimated forecast results~\ref{fig:recovery_regressions}.

\begin{figure}[h]
    \includegraphics*[width=4cm, height=4cm]{../src/output/recovery-linear-regression.png}
    \includegraphics*[width=4cm, height=4cm]{../src/output/recovery_isotonic_regression.png}
    \caption{Recovery Rate Linear and Isotonic Regressions}
    \label{fig:recovery_regressions}
\end{figure}

Both models have similar results with small deviation.

Models show that recovery trend is growing up.

% todo: describe algorithm in details

\begin{figure}[h]
    \includegraphics*[width=4cm, height=4cm]{../src/output/death_linear-regression.png}
    \includegraphics*[width=4cm, height=4cm]{../src/output/death_isotonic_regression.png}
    \caption{Death Rate Linear and Isotonic Regressions}
    \label{fig:death_regressions}
\end{figure}

Both functions strives for negative value and trend is death negative~\ref{fig:death_regressions}.

It can be concluded that trend is positive based on that

\begin{itemize}
    \item death rate is going down~\ref{fig:death_regressions}
    \item recovery rate is going up~\ref{fig:recovery_regressions}
    \item population rate is going up~\ref{fig:pop_death_corr}
\end{itemize}



% CLASSIFICATION
\subsection{Data Classification}
\label{subsec:data-classification}

Data~\ref{tab:list_of_raw_properties} was grouped by state with total count of 26 states~\ref{tab:classified_states}.

Each state was classified by mean feature~\cite{wiki-mean} for death and recovery attributes as threshold to identify states with inflated rates.

Death feature classification results: 17 state are mean and 9 are overmean~\ref{tab:classified_states}.

Recovery feature classification results: 15 state are mean and 11 are overmean~\ref{tab:classified_states}.

\begin{figure}[h]
    \includegraphics*[width=\linewidth, keepaspectratio]{../src/output/states_death.png}
    \caption{States Map By Death Attribute}
    \label{fig:states_death}
\end{figure}

New York state which is located in the north-east of the map has the highest death rate density between among states~\ref{fig:states_death}.

\begin{figure}[h]
    \includegraphics*[width=\linewidth, keepaspectratio]{../src/output/states_recovery.png}
    \caption{States Map By Recovery Attribute}
    \label{fig:states_recovery}
\end{figure}

Illinois state which is located in the middle of the map has the highest recovery rate~\ref{fig:states_recovery}.



% CLUSTERING
\subsection{Clustering}
\label{subsec:clustering}

Total \textit{2} sub-clusters have been generated based on k-means algorithms using unsupervised learning~\ref{fig:states_cluster}.

The k-mean algorithm experienced error when more than 5 clusters were calculated.

% todo: describe k-means algorithm in details

\begin{figure}[h]
    \includegraphics*[width=\linewidth, keepaspectratio]{../src/output/cluster.png}
    \caption{Cluster}
    \label{fig:states_cluster}
\end{figure}




% Conclusion
\section{Conclusion}
\label{sec:conclusion}

Number of positive cases compared to death rate has positive impact on recovery tendency and do not affect overall population tendency of USA\@.

However, COVID-19 is dangerous decease could affect anyone.

All groups of people could be infected with virus - the president of USA was diagnosed positive with COVID-19  approximately 10/01/2020 ~\cite{twitter-1311892190680014849}.

\enquote{Anyone who was a direct contact of President Trump or known COVID-19 cases needs to quarantine and should get tested ~\cite{mn-100220}}.

\enquote{More than a dozen people in President Donald Trump's circle -- including Trump himself -- have recently tested positive for Covid-19 ~\cite{cnn-20201004}}.

The research does not exclude a small possibility of a second wave of coronavirus under new abbreviation of COVID-20 in 2021.

COVID-19 had impact on business.

Many companies claim Chapter 11 due to COVID-19 impact on business. \enquote{For some, the pandemic was a chance to open a new chapter. But for many businesses, the swift and stark economic shutdown led straight to Chapter 11.  ~\cite{fortune-20200804}} $\blacksquare$.



